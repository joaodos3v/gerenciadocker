% DESCRIÇÃO DO PROBLEMA-------------------------------------------------------------------

\chapter{DESCRIÇÃO DO PROBLEMA}
\label{chap:descricao_do_problema}

Em virtude da dinamicidade e agilidade necessárias em tarefas comuns para uma empresa de \software{}, tal como a disponibilização de aplicações para clientes ou mesmo a configuração de ambiente para novos colaboradores na equipe de desenvolvimento, criou-se o conceito técnico de \textit{conteinerização}. De modo resumido, esse conceito pode ser descrito, segundo \cite{Fernandes18}, como \aspas{o processo de distribuir uma aplicação de \software{} de maneira compartimentada, portátil e autossuficiente}. Isto é, uma forma de criar um ambiente completo de qualquer aplicação desenvolvida e \aspas{empacotá-lo}, para posteriormente distribuí-lo e utilizá-lo. 

Dentro desse cenário e tendo em vista os diversos benefícios que essa prática traz aos seus utilizadores, diversas ferramentas foram criadas. No entanto, um \software{} em específico acabou destacando-se como o mais utilizado quando deseja-se implantar essa tecnologia. O \textbf{\docker{}} permite o gerenciamento completo de todos os \containers{} criados e, devido às suas funcionalidades, ganhou notoriedade na comunidade. 

No entanto, a utilização diária dessa ferramenta, geralmente realizada através de um terminal, pode se tornar uma tarefa desnecessária e até mesmo complicada, principalmente para um profissional iniciante, pois seu ambiente pode conter diversas especificidades (tal como vários \containers{} executando em paralelo) que, especialmente nos primeiros contatos com a ferramenta, podem ser difíceis de serem implementadas.

Além disso, vale ressaltar que, se necessário, o usuário deve controlar a rede (\dockerNetwork{}) em que os \containers{} estão sendo executados, o que acaba gerando ainda mais dificuldades. Com isso, fica nítido que, apesar de ser um recurso que proporciona inúmeras vantagens aos usuários, ainda existe uma barreira de adoção à essa tecnologia, principalmente em um contexto organizacional, pois o profissional que se dispõe a aprender essa nova ferramenta, precisará conciliar esse aprendizado com a realização de todas as demais atividades tradicionais que ele já é encarregado atualmente.





% JUSTIFICATIVA-------------------------------------------------------------------
\section{Justificativa}
\label{sec:justificativa}

Baseando-se nos motivos descritos no capítulo \ref{chap:descricao_do_problema} e com o desejo de aprender mais sobre essa moderna ferramenta, o grupo considerou que um monitor \web{} que abstraísse essas dificuldades de gerenciamento em uma interface intuitiva e amigável para o usuário seria, além de uma ferramenta útil para os profissionais que se enquadram nessa situação, um bom assunto para ser o tema deste projeto.