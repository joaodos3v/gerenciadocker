% ---------------------------------------------------------------------------------------------------------------
% TEMPLATE EM LATEX PARA TRABALHO DE CONCLUSÃO DE CURSO DA URI ERECHIM
% (ESTE TEMPLATE NÃO É UM PROJETO OFICIAL DA URI) CONSULTE SEU ORIETADOR CASO QUEIRA UTILIZA-LO
% Este template foi baseado no template da Universidade Tecnológica Federal do Paraná UTFPR
%
% Template baseado no projeto: http://tcc.tsi.gp.utfpr.edu.br/paginas/modelos-latex-da-utfpr
%
%----------------------------------------------------------------------------------------------------------------
% Codificação: UTF-8
% LaTeX:  abnTeX2
% ---------------------------------------------------------------------------------------------------------------

% CARREGA CLASSE PERSONALIZADA COM AS NORMAS DA URI-----------------------------------------------------------
\documentclass[oneside]{configs/uri-abntex2} %oneside -> impressão apenas frente

% INCLUI ARQUIVOS DE CONFIGURAÇÕES-------------------------------------------------------------------------------
\include{configs/pacotes}
\include{configs/configuracoes-pdf}
%%-----------------------------------------------------------------------------
%% VARIÁVEIS
%%-----------------------------------------------------------------------------
%% Utilize este arquivo para colocar valores que serão útes durante todo o
%%  documento (tal como a versão de um sistema ou o nome de uma ferramenta).

% Contêiner(singular, itálico)
\newcommand{\conteiner}{\textit{contêiner}}

% Containers (plural, itálico)
\newcommand{\containers}{\textit{containers}}

% Docker (itálico)
\newcommand{\docker}{\textit{docker}}

% Docker Network (itálico)
\newcommand{\dockerNetwork}{\textit{docker network}}

% Software (itálico)
\newcommand{\software}{\textit{software}}

% Web (itálico)
\newcommand{\web}{\textit{web}}





%%-----------------------------------------------------------------------------
%% COMANDOS
%%-----------------------------------------------------------------------------
%% Os comandos abaixo foram úteis em alguma situação e, para facilitar, foram
%%  colocados neste arquivo para centralizar sua definição.

% Insere aspas duplas acerca de um texto qualquer
\newcommand{\aspas}[1]{``#1''}


% INCLUI ARQUIVOS DO DESENVOLVIMENTO DO DOCUMENTO (PRÉ-TEXTUAIS, TEXTUAIS, PÓS-TEXTUAIS)-----------------------

% INSERE CAPA
\include{content/pre-textuais/capa}



% <START>
%   Conteúdo do Documento: base teórica.
% </START>
\begin{document}

    \pretextual
    \imprimircapa                                              	            % Comando para imprimir Capa
    
    % INSERE ELEMENTOS PRÉ-TEXTUAIS
    \include{content/pre-textuais/sumario}               			        % Sumário
    \include{content/pre-textuais/lista-figuras}           			        % Lista de Figuras

    \textual
    
    % INSERE ELEMENTOS TEXTUAIS
    % INTRODUÇÃO-------------------------------------------------------------------

\chapter{INTRODUÇÃO}
\label{chap:introducao}

\textbf{Introdução} \textit{aqui}.



\section{Seção 2}
\label{sec:secao_2}

Outro Exemplo.

                		            % Introdução
    % DESCRIÇÃO DO PROBLEMA-------------------------------------------------------------------

\chapter{DESCRIÇÃO DO PROBLEMA}
\label{chap:descricao_do_problema}

Em virtude da dinamicidade e agilidade necessárias em tarefas comuns para uma empresa de \software{}, tal como a disponibilização de aplicações para clientes ou mesmo a configuração de ambiente para novos colaboradores na equipe de desenvolvimento, criou-se o conceito técnico de \textit{conteinerização}. De modo resumido, esse conceito pode ser descrito, segundo \cite{Fernandes18}, como \aspas{o processo de distribuir uma aplicação de \software{} de maneira compartimentada, portátil e autossuficiente}. Isto é, uma forma de criar um ambiente completo de qualquer aplicação desenvolvida e \aspas{empacotá-lo}, para posteriormente distribuí-lo e utilizá-lo. 

Dentro desse cenário e tendo em vista os diversos benefícios que essa prática traz aos seus utilizadores, diversas ferramentas foram criadas. No entanto, um \software{} em específico acabou destacando-se como o mais utilizado quando deseja-se implantar essa tecnologia. O \textbf{\docker{}} permite o gerenciamento completo de todos os \containers{} criados e, devido às suas funcionalidades, ganhou notoriedade na comunidade. 

No entanto, a utilização diária dessa ferramenta, geralmente realizada através de um terminal, pode se tornar uma tarefa desnecessária e até mesmo complicada, principalmente para um profissional iniciante, pois seu ambiente pode conter diversas especificidades (tal como vários \containers{} executando em paralelo) que, especialmente nos primeiros contatos com a ferramenta, podem ser difíceis de serem implementadas.

Além disso, vale ressaltar que, se necessário, o usuário deve controlar a rede (\dockerNetwork{}) em que os \containers{} estão sendo executados, o que acaba gerando ainda mais dificuldades. Com isso, fica nítido que, apesar de ser um recurso que proporciona inúmeras vantagens aos usuários, ainda existe uma barreira de adoção à essa tecnologia, principalmente em um contexto organizacional, pois o profissional que se dispõe a aprender essa nova ferramenta, precisará conciliar esse aprendizado com a realização de todas as demais atividades tradicionais que ele já é encarregado atualmente.





% JUSTIFICATIVA-------------------------------------------------------------------
\section{Justificativa}
\label{sec:justificativa}

Baseando-se nos motivos descritos no capítulo \ref{chap:descricao_do_problema} e com o desejo de aprender mais sobre essa moderna ferramenta, o grupo considerou que um monitor \web{} que abstraísse essas dificuldades de gerenciamento em uma interface intuitiva e amigável para o usuário seria, além de uma ferramenta útil para os profissionais que se enquadram nessa situação, um bom assunto para ser o tema deste projeto.          		            % Descrição do Problema
    % OBJETIVOS-------------------------------------------------------------------

\chapter{OBJETIVOS}
\label{chap:objetivos}




\section{Objetivo Geral}
\label{sec:objetivo_geral}

\begin{itemize}
    \item Criar uma ferramenta que possibilite gerenciar os \containers{} em execução na máquina
\end{itemize}




\section{Objetivos Específicos}
\label{sec:objetivos_especificos}

\begin{itemize}
    \item Tornar a ferramenta flexível, permitindo que o usuário escolha o sistema operacional do \conteiner{} (com 4 opções)
    \item Construir uma interface amigável e intuitiva para o usuário
    \item Executar corretamente o algoritmo (\adaptive{}), que detectará falha nos \containers{}
    \item Criar um aplicativo complementar à solução, que será responsável por disparar uma notificação sempre que uma falha ocorrer e trará mais mobilidade para o usuário
\end{itemize}                  		            % Objetivos
    % DESCRIÇÃO DA IMPLEMENTAÇÃO-------------------------------------------------------------------

\chapter{DESCRIÇÃO DA IMPLEMENTAÇÃO}
\label{chap:descricao_da_implementacao}

% ADAPTIVE DSD-------------------------------------------------------------------
\section{Adaptive DSD}
\label{sec:adaptiveDSD}

O Adaptive DSD (\textit{Adaptive Distributed System-Level Diagnosis}) é um algoritmo para diagnóstico em redes completamente conectadas. Onde seu funcionamento é ao mesmo tempo 
adaptativo e distribuído. Foi desenvolvido para que cada máquina que possua o algoritmo em execução possa realizar o teste e também ser testada por outras máquinas na rede.
É caracterizado como adaptativo por não depender e nem restrigir o número de máquinas na rede, necessitando, no mínimo uma máquina para o teste. Para a execução dos testes, não é levado
em consideração falhas na rede, pois o objetivo deste algoritmo é, testar o processamento ou funcionamento específico de um processo na máquina.

\subsection{Funcionamento}
\label{sub:adaptiveDSD_Funcionamento}
O algoritmo possui duas listas, que possuem de tamanho o número de máquina conectadas à rede, as listas são: o vetor TESTED\_UP, que irá guardar na posição da máquina atual, qual 
o índice da máquina testada que possui funcionamento normal; o vetor STATE, que armazena o estado das máquinas, tendo inicialmente o valor FALHO para todas e, caso uma máquina tenha seu 
funcionamento confirmado, esta receberá o valor NORMAL no vetor. A cada rodada o os vetores são atualizados e enviados às outros máquinas na rede.

Na primeira rodada, uma máquina irá iniciar o teste seguindo a lista de máquina existentes e disponíveis na rede. Esta máquina irá percorrer a
lista de máquinas fará uma requisição de teste à próxima máquina da lista. Caso a máquina a ser testada retornar uma resposta de funcionamento correto, a máquina que está realizando o 
teste recebe os dados e envia à máquina testada, que por sua vez irá executar o mesmo processo com a máquina seguinte, até que todas as máquinas tenham sido testadas. Por outro lado, se 
a máquina testada, retornar algum erro, será marcada como falha, e a máquina que está testando irá testar a próxima máquina da lista, até encontrar outro máquina com funcionamento normal 
ou até que a lista de máquinas disponíveis acabe.

A segunda rodada será para atualizar as informações de todas as máquinas na rede sobre o estado de funcionamento de cada máquina. Inicialmente a primeira máquina com funcionamento normal 
irá verificar na lista de máquinas, qual a próxima máquina funcionando, e irá enviar os dados da rede. Ao receber os dados da rede, a máquina receptora irá prosseguir com a distrbuição 
de informações.
    % CONCLUSÃO - SkillBoard -------------------------------------------------------------------
\chapter{CONCLUSÃO}
\label{chap:conclusao}

Conclusão aqui.                   		            % Conclusão

    \postextual

    % INSERE ELEMENTOS PÓS-TEXTUAIS
    % REFERÊNCIAS------------------------------------------------------------------

% Carrega o arquivo "referencias.bib" e extrai automaticamente as referências citadas
{\renewcommand{\contentsname}{MAC}}
\bibliography{./gerenciadocker}
\bibliographystyle{abntex2-alf} % Define o estilo ABNT para formatar a lista de referências
% OBSERVAÇÕES------------------------------------------------------------------
% Este arquivo não precisa ser alterado.           			        % Referências


\end{document}
% <ENCERRA>
%   Conteúdo do Documento: base teórica.
% </ENCERRA>